\documentclass{article}
\usepackage{enumitem}
\usepackage{listings}
\usepackage{amsfonts}
\usepackage{latexsym}
\usepackage{fullpage}
\usepackage{graphicx}
\usepackage{tikz-timing}
\usepackage{tabto}
\usepackage{listings}
\usepackage[utf8]{inputenc}
\usepackage[portuguese]{babel}
\usepackage{amssymb}

\begin{document}

\begin{flushleft}
\textbf{\Large{Sistemas Distribuídos - Trabalho 3}}\\
\vspace*{1cm}
\textbf{\large{Clara Szwarcman}}\\
\vspace*{0.25cm}
\textbf{\large{Guilherme Simas}}\\
\end{flushleft}


\section*{Funcionamento do Programa}

	\tab Cada nó possuirá as seguintes informações: um vetor de vizinhos; uma flag para cada tipo de evento, que indica se aquele nó possui uma source para aquele evento; apenas uma possível source para cada tipo de evento (se uma nova source for detectada, apenas a com o menor número de hops prevalece); uma flag para cada tipo de evento que indica se o nó possui alguém interessado naquele evento; uma flag para cada tipo de evento que indica se aquele nó está aguardando a resposta para um request. A estrutura de fila será utilizada para armazenar interesses.\\

	O primeiro passo é o reconhecimento dos vizinhos. Para isso cada um dos nós manda um broadcast a cada 100 milisegundos por 5 segundos. O mesmo nó recebe o broadcast de seus vizinhos e armazena os ids no vetor neighbours. Quando chega uma nova mensagem, é checado se aquele id já está no vetor, se não estiver o novo vizinho é adicionado. O vetor tem tamanho fixo 8, que é o número máximo de vizinhos presente na arquitetura do simulador.\\
	
		A segunda parte possui dois loops ocorrendo em paralelo. Um que detecta um novo evento ou um novo interesse no próprio nó e outro que aguarda o recebimento e tratamento de uma mensagem. Uma mensagem pode possuir tipos 6 diferentes pré estabelecidos: SOURCE\_MIN, SOURCE\_MAX, INTEREST\_MIN, INTEREST\_MAX, REQUEST, INFO. \\
		
		O loop que detecta um evento começa requisitando a temperatura, e caso ela esteja abaixo ou acima de limites definidos uma flag é setada para que uma mensagem seja enviada. A mensagem terá o tipo "SOURCE\_MIN" ou "SOURCE\_MAX" e terá como target um vizinho randômico. Quando surge um evento o nó também seta a si próprio como source, e a temperatura lida como informação. O loop de evento é análogo, utilizando o sensor de luz. Quando um interesse é gerado uma flag é setada, indicando que aquele nó está aguardando a resposta para um evento.\\
		
		O loop que trata as mensagens recebidas foi estruturado tentando generalizar ao máximo esse tratamento. Primeiro ele aguarda a chegada de uma mensagem. Quando uma nova mensagem é recebida, uma mensagem "padrão"(target como vizinho randômico, tipo igual ao tipo recebido,hops igual a hops recebidos acrescido de 1) é montada. Os parâmetros dessa mensagem serão alterados de acordo com as particularidades de cada tipo, e, no final do loop, essa mensagem será enviada caso a flag correspondente esteja setada como \textit{true}.\\		
		
		 Ao receber uma mensagem do tipo SOURCE, caso a flag para source daquele tipo de evento seja \textit{false}, ela é trocada para \textit{true} e o id do nó source armazenado. Caso seja \textit{true}, a source com o menor número de hops é mantida. A flag de envio é setada como false caso o número de hops ultrapasse o limite, caso contrário a mensagem será repassada para um vizinho aleatório.\\
		 
		Em uma mensagem do tipo INTEREST é criada uma instância do tipo \textit{interestInfo}, que contém o tipo do interesse e o id do nó interessado. Essa instância é armazenada na fila. A mensagem é reenviada com o memso critério da SOURCE\\
		
		Ao receber uma REQUEST, além de colocar a mesma variável do tipo \textit{interestInfo} na fila (ou seja, um nó que realiza uma requisição é tratado como um nó interessado naquele tipo de informação), o nó checa se ele possui aquela informação e, em caso positivo, ele dispara uma nova mensagem com o tipo INFO.\\
		
		No tipo INFO o nó faz um loop do tamalho da fila (qSize()). A cada iteração um elemento da fila é retirado, e se o tipo de evento daquele elemento for igual ao da informação que chegou, a mensagem de INFO é enviada para aquele nó. Se os tipos forem diferentes, o nó é colocado novamente na fila, para aguardar o tipo de informação correto.\\
		
		Ao fim desse comportamento particular de cada tipo de mensagem, o nó checa se a flag de envio de mensagem está setada como \textit{true} e encaminha a mensagem em caso positivo. Por fim, o nó checa através de flags se ele possui pelo menos uma source e um interesse no mesmo tipo de evento e ainda não está no aguardo por uma informação daquele mesmo tipo. Caso essas três condições sejam verdadeiras, ele dispara uma nova request para o nó source.\\
		
\section{Uso da Aplica\'{c}\~{a}o}
	\tab Ao inicializar, a aplica\'{c}\~{a}o ir\'{a} entrar\'{a} em um processo de incializa\'{c}\~{a}o onde os vizinhos ir\~{a}o se identificar. Recomenda-se que aguarde o final desse processo (marcado pela falta de mensagens circulando pela rede).\\
		Para gerar um agente de SOURCE ou INTEREST, deve-se subir ou descer os sensores de temperatura e luminosidade (temperatura para SOURCE e luminosidade para INTEREST). Os valores para cada sensor devem atingir mais que 50 unidades do valor inicial para que aconte\'{c}a este evento. Caso o valor seja abaixo do inicial, os agentes emitidos ser\~{a}o referentes a eventos de temperatura m\'{i}nima. Caso sejam maiores, ser\~{a}o referentes a eventos de temperatura m\'{i}xima. A checagem \'{e} feita a cada 5 segundos.\\
		Ao receber um REQUEST, o LED0 acender\'{a}. Ao receber uma INFO, o LED1 acender\'{a}. Finalmente, quando a informa\'{c}\~{a}o chega ao n\'{o} que gerou o interesse, o LED3 acender\'{a}.
		
		
\section{Dificuldades}

	\tab As maiores dificuldades vieram do uso de Terra. Além de todo o debug ser feito utilizando apenas 3 leds, há pouquíssima documentação e informação das funções disponíveis e o que elas fazem. As APIS de fila (queue) tamb\'{e}m apresentam esse mesmo problema, al\'{em} de n\~{a}o acompanharem quaisquer exemplos.\\
		O ambiente de desenvolvimento em si tamb\'{em} dificulta, visto que as ferramentas s\'{o} se encontram dispon\'{i}veis em uma m\'{a}quina virtual de recursos limitados, o que torna todo o processo lento.


		
\end{document}